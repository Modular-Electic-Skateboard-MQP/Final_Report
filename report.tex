\documentclass[a4paper,10pt]{article}
\usepackage[utf8]{inputenc}

%opening
\title{Modular Electric Skateboard}
\author{Nick Fajardo, Phillip Konyeaso, Zane Weissman}

\begin{document}

\maketitle

\begin{abstract}
The Modular Electric Skateboard (MESB) is an electric skateboard (ESB) that allows the user to mix and match components tailored to a specific user’s needs. The MESB is equipped with a TBD watt motor controlled by a handheld RC controller and capable of acheiving speeds as high as TBD. The deck, the part of the board on which the user stands while riding, can be disassembled and reassembled by hand to swap out major mechanical components of the board. The board's electrical system includes rails for mounting hot-swappable accessories.

\end{abstract}

\section{Background}
In order to commute from one place to another in an urban environment, one has to consider multiple options in order to get around without a significant time delay in their route. Solutions exist, such as bicycles, busses, and trains. However, as of recently, there has been a growing market in a specific solution, electric skateboards. In order to understand this market, one should be knowledgeable on skateboard parts, electric skateboard parts, and available ESBs for consumers.
\subsection{Skateboard Parts and Terminology}
Skateboarding has been around since the mid 1900’s, and has changed drastically since its introduction. The diagrams on the right below show the different parts that make a modern complete skateboard, and the list below defines each part in detail.
\begin{itemize}
\item Griptape - Main method of retaining a surface that has plenty of friction between the rider and the skateboard. Comes in different grits (measure of “roughness”) and colors
\item Deck - The plate where the trucks and griptape is attached. It is essential for any deck to be reliable, but also have some amount of flex for comfort. Varies in size and material.
\begin{itemize}
\item Nose - Front of the board. A typical skateboard has a nose that is angled up
\item Tail - rear of the board. A typical skateboard has a tail that is angled up
\end{itemize}
\item Trucks - The assembly that holds all the parts needed to have non powered, fixed wheels below the deck.
\begin{itemize}
\item Riser Pad - pad used to add distance between the wheel and the deck. Often made of a material conducive to dampening vibration, such as rubber or cork.
\item Base Plate - metal piece used for the rest of the truck assembly to be attached to. Has a threaded rod sticking out of it for the assembly to align and attach to.
\item Bushings - Spacers between the Kingpin, the hanger, and the base plate. Bushings are used to adjust the force required to turn while riding. I.E., a harder to compress bushing will require the user to lean to a direction more in order to perform the same degree of turn as a softer to compress bushing. Often made of some sort of rubber polymer
\item Kingpin - Nut used to cap the hanger and bushing assembly. Most often a locknut.
\item Axel (and Axle Nut) - rod used to attach the wheels.
\item Bearings - device inserted into the wheels in order for the wheels to rotate smoothly
\item Wheels - device that allows the skateboard to move, varies in size, shape, and color.
\end{itemize}
\item Hardware - term for the nuts and bolts needed to attach the base plate to the deck.
\item Longboard - Type of skateboard often used for traveling long distances. Has an elongated deck and often is more flexible than a skateboard.
\end{itemize}

\subsection{Electric Skateboards}
Electric Skateboards are a variant of skateboards that attach a motor and gearing system to either the back and/or front of the board. Most ESBs on the market today are tailored to provide a device that can handle a moderate commute, while being lite and portable. Hence, most often than not adopting a longboard and adding a drive system that consists of motors with a battery. 
One example of an ESB is the Boosted Board brand “Dual+ XR” pictured on the right, which happens to also be one of the most popular off the shelf ESBs on the market today. It retails for \$1,599 and claims a top speed of 22 mph and a range of 14 miles with its 2000W motor and regenerative braking. Some more examples of popular ESBs can be found in the table below:

\begin{tabular}{l l l l l}
Board Name & Top Speed (mph) & Battery Range (Miles) & Weight (lbs) & Price (Dollars) \\\hline
Inboard M1 & 22 & 7 & 14.5 & 1,000 \\
Boosted Plus/Stealth & 22/24 & 14 & 17 & 1,399/1.599 \\
Blink Qu4tro & 23 & 22 & 24 & 1,699 \\
Boosted Mini S/X & 18/20 & 7/14 & 15/16.8 & 749/999 \\
Onewheel +/+XR & 19 & 5-7/12-18 & 25 & 1,399/1,799 \\
\end{tabular}

\section{System Overview}
\section{Design Analysis}

\end{document}
